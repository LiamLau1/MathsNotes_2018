\documentclass{report}
\usepackage{amsmath}
\usepackage{geometry}
\usepackage{amssymb}
 \geometry{
 a4paper,
 total={170mm,257mm},
 left=20mm,
 top=20mm,
 }
\title{\fontsize{40}{48} \selectfont \textbf{Mathematical Notes \\  2018}}
\author{\textbf{L. L.H. Lau}}
\date{}


\begin{document}
\maketitle
\chapter*{Groups}
\vspace{10.0mm}
\section*{Group Axioms}
\large{ A group G is a set G with a binary operation denoted by $\bullet$  which satisfies the following axioms:\\ 
\begin{enumerate}
	\item \textbf{Identity}\\
		\linebreak
		$\forall$ a $\in$ G, $\exists$ e $\in$ G, s.t. e $\bullet$ a = a $\bullet$ e = a   
	\item \textbf{Inverse}\\
		\linebreak
		 $\forall$ a $\in$ G, $\exists$ $a^{-1}$ $\in$ G, s.t. $a^{-1}$ $ \bullet$ a = a $\bullet$ $a^{-1}$ = e
	\item \textbf{Associativity}\\
		\linebreak
		$\forall$ a, b, c $\in$ G, a $\bullet$ (b $\bullet$ c) = (a $\bullet$ b) $\bullet$ c
	\item \textbf{Closure}\\
		\linebreak
		$\forall$ a, b $\in$ G, (a $\bullet$ b) $\in$ G	
\end{enumerate}


}

\section*{Order}
\large{
	The order of a group G is denoted $\|$ G $\|$ and gives the number of elements in G. 
\\	The order of an element g $\in$ G, denoted by n, is the smallest integer n such that $g^{n}$ = e.
	It is trivial to see that ord(e) = 1 and if an element t $\in$ G is self inverse, ord(t) = 2.

\subsection*{Proof that the order of g \& $g^{-1}$ are the same}
	Let $n_1$ be s.t. $g^{n_1}$ = e and $n_1$ is the smallest such integer.
	\begin{alignat*}{2}
		&(g^{-1}g) &&= e \\
		\implies &(g^{-1}g)^{n_1} &&= e^{n_1} = e\\
		\implies &(g^{-1})^{n_1}g^{n_1} &&= e\\ 
		\implies &(g^{-1})^{n_1} &&= e
	\end{alignat*}
	Here $n_1$ is a multiple of the smallest such integer for $g^{-1}$. Similarly we can define the smallest integer $n_2$ 
	which satisfies $(g^{-1})^{n_2} = e$ . By symmetry we get $n_1$ and $n_2$ being multiples of each other. $\therefore n_1 = n_{2}$
	\newpage

\subsection*{Exercises \\}
	\begin{enumerate}
	\item Proof of the uniqueness of the identity and the inverse.\\
	\item Prove that  $\forall$ a, b $\in G$ $ord(ba) = ord(ab)$.
	\end{enumerate}

}

\section*{Types of Groups}
\large{
	\subsection*{Abelian Groups}
		A group G is abelian if it follows the group axioms and also has a commutativity axiom. Where $\forall$ a, b $\in G$ $a \bullet b = b \bullet a$ .  
	\subsection*{Cyclic Group}
		For a cyclic group G of order n, all emements of the group can be generated from one element of the group, e.g. X.
		\begin{equation*}
			G = \{I, X, X^2, X^3, ... , X^{n-1}\}
		\end{equation*}
		A cyclic group is abelian and all elements of G, apart from I have order n, the same as the ord(G) - the number of elements in group G.
	\section*{Subgroup}
		A group H is a subgroup of G, denoted by $H \leq G$, \textbf{iff} the set H is a subset of the set G, denoted by $H \subseteq G$, and follows the group axioms.
		This can be summarised by the following, where the group H is a non empty set with a binary operator $\bullet$ : 
		\begin{equation*}
			\forall a, b \in H, (ab^{-1})\in H	
		\end{equation*}
		The logic follows as such:
		\begin{alignat*}{2}
			& a \in H \\
			&aa^{-1}\in H &&\implies e \in H \text{ (By substitution of b for a- identity axiom)} \\
			&ea^{-1}\in H &&\implies a^{-1} \in H \text{(Inverse Axiom)}\\ 
			&b \in H &&\implies b^{-1} \in H \therefore a(b^{-1})^{-1} \in H \implies ab \in H \text{(Closure)}
		\end{alignat*}
}

\chapter*{List of Common DEs} 
\vspace{10.0mm}
\section*{First Order ODEs}
\large{
	\subsection*{\textbf{Separable:}}
	 	$\frac{dy}{dx} = f(x)g(y)$ \\ \\ If you can't do this, just give up.

	\subsection*{\textbf{Exact:}} $A(x,y)dx + B(x,y)dy = 0$ and $\frac{\partial A}{\partial y} = \frac{\partial B}{\partial x}$ \\ \\This follows from the symmetry of mixed partial derivatives.
\\ To find the function U which satisfies $\frac{\partial U}{\partial x} = A$ and $\frac{\partial U}{\partial y} = B$ we can take the union of the generalised functions solved by integrating. Remember due to the partial derivatives we get some functions of x or y instead of constants of integration.

	\subsection*{\textbf{Inexact:}} $A(x,y)dx + B(x,y)dy = 0$ and $\frac{\partial A}{\partial y} \neq \frac{\partial B}{\partial x}$

}
\chapter*{Integral Transforms}
\vspace{10.0mm}
\section*{Brief Recap on Fourier Series}
\large{
	For a function $f(x)$ to be represented by a Fourier Series, it must satisfy the Dirchlet conditions:
	\begin{enumerate}
			\item The function must be periodic.
			\item It must be single valued and continuous, except possibly at a finite number of finite discontinuities.
			\item It must have a finite number of minima and maxima within one period, unlike $sin(1/x)$
			\item The integral over one period of $|f(x)|$ must converge.
	\end{enumerate}

	We can represent a function $f(x)$ as such
	\begin{equation*}
		f(x) = \frac{a_{0}}{2} + \sum_{n = 1}^{\infty}a_{n}cos(\frac{n\pi x}{L}) + b_{n}sin(\frac{n\pi x}{L})
	\end{equation*}

}
\section*{Fourier Transform and the Dirac Delta Function}
\large{
	The FT takes the time period T to the limit of infinity, so that $\omega$ becomes a continuous variable $\omega = \frac{2\pi}{T}$ and $\Delta\omega$ becomes vanishingly small. Hence the infinite sum for the complex Fourier Series becomes an integral. Look at Mathematical Notes 2018 for the "derivation".\\
	We define the forward and inverse transforms as such:
	\begin{alignat*}{2}
		& \hat{f}(\omega) = \frac{1}{\sqrt{2\pi}}\int_{-\infty}^{\infty}f(t)e^{-i\omega t}dt \qquad \text{(forward)}\\
		& f(t) = \frac{1}{\sqrt{2\pi}}\int_{-\infty}^{\infty}\hat{f}(\omega)e^{i\omega t}d\omega \qquad \text{(inverse)}
	\end{alignat*}
	The forward transform takes the function f from time space to frequency space, and the inverse does the reverse. The FT exists provided that the integral of $|f(x)|$ from $-\infty$ to $\infty$ and that any discontinuities in f(x) are finite.
}


\end{document}
